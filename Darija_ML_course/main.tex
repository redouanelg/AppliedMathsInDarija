\documentclass[16pt]{beamer}
\usetheme{Warsaw}
\usepackage{enumitem}
\usepackage{fontspec}
\usepackage{color}
\usepackage{polyglossia}
\setbeamertemplate{frametitle}[default][right]% align the frametitle to the right
\setdefaultlanguage{english}
\setotherlanguage[calendar=gregorian,numerals=maghrib]{arabic}
\newfontfamily\arabicfont[Script=Arabic,Scale=1]{Amiri}
\newfontfamily\arabicfontsf[Script=Arabic,Scale=1]{Amiri}

%~~~~~~~~~~~~~~~~define bullet for enumerate~~~~~~~~~~~~~

\newcommand{\myenum}[1]
{
 \par\vspace{3pt}\LR{\hspace*{3pt}%
  \begin{pgfpicture}{-1ex}{-0.65ex}{1ex}{1ex}
    \usebeamercolor[fg]{item projected}
    {\pgftransformscale{1.8}\pgftext{\normalsize \pgfuseshading{bigsphere}}}
    {\pgftransformshift{\pgfpoint{0pt}{0.5pt}}
      \pgftext{\usebeamerfont*{item projected}\textcolor{white}{#1}}}
  \end{pgfpicture}%
  \hspace{1pt}%
}}

%~~~~~~~~~~~~~~~~~~define bullet for itemize~~~~~~~~~~~~~~~

\newcommand{\myitem}{\par\vspace{3pt}\hspace{5mm}\LR{\hspace*{3pt}%
  \begin{pgfpicture}{-1ex}{-0.65ex}{1ex}{1ex}
    \usebeamercolor[fg]{item projected}
    {\pgftransformscale{1.1}\pgftext{\normalsize\pgfuseshading{bigsphere}}}
      \end{pgfpicture}%
  \hspace{1pt}%
}}

%~~~~~~~~~~~~~~~~~~~~~~~~~~~~~~~~~~~~~~~~~~~~~~~~~~~~~~~~~
\logo{\includegraphics[width=1.4cm]{ttr.png}}
\title{\textarabic{تويشيات فالتعلم الآلي}}
\author{Redouane Lguensat\\
\texttt{fb.com/AppliedMathsInDarija}}

\date{February 2017}
\institute{Simple introduction to Machine Learning in Moroccan Arabic}
%-----------------------------------------------
\begin{document}

  \begin{frame}
    \titlepage
  \end{frame}
  
\begin{frame}
\frametitle{\textarabic{السلام عليكم}}
\begin{minipage}{0.95\textwidth}
\begin{Arabic}

\begin{enumerate}[label=\protect\myenum{\arabic*}]
\item هاد التقديم محاولة متواضعة باش نوصل بعض الافكار على التعلم الآلي بشكل مبسط و بالدارجة المغربية

\item هاد التقديم الاول عايكون بداية ان شاء الله لسلسلة تعريفية بأهمية التعلم الآلي و تطبيقاته 

\item  الناس اللي باغين يقراو التعلم الآلي خاصهم يتعمقو اكثر فالدروس المتوفرة على الانترنت، هادشي اللي كاندير عا ديك الدغمة الاولى لتحبيب هاذ المجال للتلاميذ و الطلبة المغاربة

\end{enumerate}    
\end{Arabic}
\end{minipage}
\end{frame}
 
\begin{frame}
\frametitle{\textarabic{شناهو التعلم الآلي؟}}
\begin{minipage}{0.95\textwidth}
\begin{Arabic}
 هو مجموعة خوارزميات Algorithms كاتاخذ كمية "كبيرة" من البيانات Data و تستخرج منهم عدة معلومات تا كاتولي "معلمة" فيهم.
\end{Arabic}

\begin{figure}
\includegraphics[scale=0.12]{robot2}
\end{figure}
\end{minipage}

\end{frame}

\begin{frame}
\frametitle{\textarabic{عطيني مثال آ عشيري}}
\begin{minipage}{0.95\textwidth}
\begin{Arabic}
من اهم المشاكل اللي كايحلها التعلم الآلي هي التصنيف \text{Classification}

كايكونو عندك بيانات من عدة اصناف و كايخصك شي خوارزمية تفرق ليك الاصناف اوتوماتيكيا:  عندك شي 40 ولا 400 صورة ديال اتاي و القهوة و مامساليش نتا تبقا تشوف صورة بصورة شكون اتاي و شكون القهوة 

\begin{enumerate}[label=\protect\myenum{\arabic*}]
\item  الحل الاول: جيب حميد و قوليه يبقا يشد صورة و بصورة و يحط ديال اتاي فبلاصة اتاي و ديال القهوة فبلاصة القهوة.

\end{enumerate} 

حل ماشي زوين حيث:

\begin{itemize}
\item حميد خاصك تخلصو
\item حميد بنادم و كايجيه الجوع و النعاس و.. و..
\item 
\end{itemize}

\end{Arabic}
\end{minipage}

\end{frame}


\begin{frame}
\frametitle{\textarabic{التصنيف}}
\begin{minipage}{0.95\textwidth}
\begin{Arabic}
\begin{enumerate}[label=\protect\myenum{\arabic*}]
\item الحل الثاني: عادي تاخذ خوارزمية تعلم آلي، و تعلمها الفرق بين تصاور اتاي و القهوة، عاد تعطيها تصاورك نتا. 
\end{enumerate} 

\end{Arabic}

\begin{figure}
\includegraphics[scale=0.6]{robot}
\end{figure}
\end{minipage}

\end{frame}

\begin{frame}
\frametitle{\textarabic{ مراحل استعمال خوارزميات التعلم الآلي }}
\begin{minipage}{0.95\textwidth}
\begin{Arabic}
في اي خوارزمية ديال التعلم الآلي، خاصك اولا بزآآاف ديال البيانات، اللي هي فالحالة ديالنا: بزاف ديال صور اتاي و بزاف ديال صور القهوة، نفتارضوا عندنا شي عشرات الآلاف ديال الصور من كل صنف. من بعد كانتبعوا هاذ المراحل:

\begin{itemize}
\item[$\bullet$] كانشدو تقريبا شي 80 \% باش تتعلم منها الخوارزمية (فهاد المرحلة را كانعطيوها الصورة و ايضا الصنف ديالها) و داكشي اللي بقا باش الخوارزمية تبقى تجرب فيه واش خدامة مزيان ولا لا. هاد جوج مراحل كايسماو التعلم \textcolor{blue}{Training} و التحقق \textcolor{green}{Validation}

\item[$\bullet$] من بعد عاد نقدرو نقولو را الخوارزمية واجدة و "معلمة"، و دابا عاد كانديرو التطبيق \textcolor{red}{Test} و كانعطيوها 40 ولا 400 صورة ديالنا باش تصنفهم لراسها بلا مساعدة من عندنا

\end{itemize} 
\end{Arabic}
\end{minipage}
\end{frame}


\begin{frame}
\frametitle{\textarabic{المراحل فمثال القهوة و آتاي}}
\begin{minipage}{0.95\textwidth}
 \textcolor{blue}{Training} and \textcolor{green}{Validation}
\begin{figure}
\includegraphics[scale=0.5]{m3lm}
\end{figure}
\end{minipage}
\end{frame}


\begin{frame}
\frametitle{\textarabic{المراحل فمثال القهوة و آتاي}}
\textcolor{red}{Test}
\begin{figure}
\includegraphics[scale=0.6]{atay}
\end{figure}
\end{frame}



\begin{frame}
\frametitle{\textarabic{ التصنيف فالحياة اليومية}}
\begin{minipage}{0.95\textwidth} 
\begin{Arabic}
 من اول التطبيقات فالحياة اليومية ديال التصنيف بخوارزميات التعلم الآلي كانت هيا تصنيف الارقام المكتوبة بخط اليد، كاتستعمل فقراءة الرموز البريدية اللي فالأظرفة و حتى المبالغ المالية اللي فالشيكات.
 \end{Arabic}
\begin{figure}
\includegraphics[scale=0.45]{mnist.jpg}
\end{figure}
\end{minipage}
\end{frame}

\begin{frame}
\frametitle{\textarabic{ التصنيف فالحياة اليومية}}
\begin{minipage}{0.95\textwidth} 
\begin{Arabic}
دابا را كاين ففايسبوك نيت، ملي كاتدير تحميل صورة تاعك مع صحابك، فايسبوك كايتعرف على وجوه صحابك اوتوماتيكيا (ايلا كانو كايحطو تصاور وجههم ففايسبوك)
 \end{Arabic}
\begin{figure}
\includegraphics[scale=0.7]{tag}
\end{figure}
\end{minipage}
\end{frame}

\begin{frame}
\frametitle{\textarabic{ التصنيف فالحياة اليومية}}
\begin{minipage}{0.95\textwidth} 
\begin{Arabic}
و دابا را ولا كايدخل لمجالات كالطب و المحاماة و حتى الفيزياء الفلكية... الخ... الخ... 

مثلا فالطب، كاين ابحاث جارية حالية فامكانية استعمال التصنيف بالشبكات العصبونية العميقة (واحد النوع من خوارزميات التعلم الآلي) باش من تصويرة تعرف صنف  سرطان الجلد اوتوماتيكيا.
 \end{Arabic}
\begin{figure}
\includegraphics[scale=0.15]{med}
\end{figure}
\end{minipage}
\end{frame}

\begin{frame}
\frametitle{\textarabic{يتبع..}}
\begin{minipage}{0.95\textwidth} 
\begin{Arabic}
التعلم الآلي ماكايتستعملش عا فالتصنيف، كايتستعمل فبزاف ديال التطبيقات اللي عانحاول ندوي عليهم فالتقديمات الجاية، شكرا على التتبع !

لكل التعاليق و الاقتراحات راسلوا صفحة "تويشيات فتطبيق الرياضيات" على الفايسبوك
\end{Arabic}
\begin{figure}
\includegraphics[scale=0.2]{atay2}
\end{figure}
\end{minipage}
\end{frame}


\end{document} 
